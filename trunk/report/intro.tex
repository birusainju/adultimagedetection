
\chapter{Introduction}
\pagenumbering{arabic}
\setcounter{page}{1}
Our human civilization has been influenced and intoxicated by the web revolution. People of every age use the web for different needs and purposes. Some use it for fun; some use it for their studies and find information while some live on it. Images are essentially part of the modern web and we all agree to the fact that a single picture is worth thousand words. We can find all sort of information on the web and it has been a part of our daily life. However, it also has abundance of images and contents that may be unsuitable for certain age groups. Finding pornographic images posted on social sites and links on study groups is not a new thing in today’s world. Pornographic images certainly need to be managed and unavailable to children and men at work.  

%
\section{Background}\label{intro}
Our human civilization has been influenced and intoxicated by the web revolution. People of every age use the web for different needs and purposes. Some use it for fun; some use it for their studies and find information while some live on it. Images are essentially part of the modern web and we all agree to the fact that a single picture is worth thousand words. We can find all sort of information on the web and it has been a part of our daily life. However, it also has abundance of images and contents that may be unsuitable for certain age groups. Finding pornographic images posted on social sites and links on study groups is not a new thing in today’s world. Pornographic images certainly need to be managed and unavailable to children and men at work.  
\par
At the same time, features of the current version of \LaTeX\ (\LaTeXe)
are illustrated --- such as mathematical expressions, numbering and
cross-referencing, bibliography and citations, graphics and tables.
Comparison of the source files with the printer-ready document will
answer a few FAQs: \Quote{How can I do \dots\ in \LaTeX?}.
\par
However, this is {\em not} a textbook on \LaTeX\ --- for that, use the
\lq\nss\rq\ notes by Oetiker \etal\ \cite{NSS}. They are written for
novices, and are a pleasure to read. They are available free on-line,
and are kept up-to-date. The \LaTeX\ book at \textsl{Wikipedia}
\cite{WL} includes the \nss\ material and is good for reference too.
Access these via the \LaTeX\ resources page \cite{LAT}.
\par
For more advanced features see \eg\lq\comp\rq\ \cite{MG}.
\par
Well-meant advice on \LaTeX\ for report-writing and poster-making is
available\footnote{From  \texttt{bob.johnson@dur.ac.uk}} in room CM315,
where there are reference copies of both \comp\ \cite{MG} and
\textsl{The Graphics Companion} \cite{GRM}.
\par
Even if you are misguided enough \cite{AC} to prepare your report in
\textsl{Word}, this template at least exemplifies a good structure ---
and gives advice about references and help with typography.
%
\section{Contents}\label{intro:contents}
The main body of this report is divided as follows.
\par
Chap.~\ref{sec:formulas} has some examples of mathematics, then
Chap.~\ref{sec:graphics} deals with graphics and includes
Sec.~\ref{sec:tables} about tables. The Conclusion, in
Chap.~\ref{andfinally}, summarises what's been achieved, the open
questions and what could be done next.
\par
Then comes the Bibliography, listing all sources of material, data and
computer programs used, \etc. Its construction is explained in
\cite[Sec.~4.2]{NSS} and there's more about it in App.~\ref{app:refs}.
\par
Otherwise appendices typically hold basic background theory, or
additional or similar examples, or longer proofs (App.~\ref{app:proofs})
 --- anything you need but which would hold up the main flow of the
story. You could also use an appendix for listings of any computer
programs that you've written (App.~\ref{app:programs}).
\par
Here there's information about using a PC (App.~\ref{app:pc}) plus
brief advice on grammar and typography (App.~\ref{app:typo}).
%
